\chapter*{Введение}							% Заголовок
\addcontentsline{toc}{chapter}{Введение}	% Добавляем его в оглавление
Генератор псевдослучайных чисел (ГПСЧ) — алгоритм, порождающий последовательность чисел, элементы которой почти независимы друг от друга.
Генераторы псевдослучайных чисел применяются во многих областях математики и информатики, в частности в кодировании и криптографии. Для этих задач чрезвычайно важны характеристики, которыми обладает генератор.
В данной работе приводится описание консольного приложения, позволяющего проводить анализ методом визуализации. 

Цель работы: Получить инструмент, помогающий оценить качество работы различных генераторов псевдослучайных чисел путем визуализации.

Для достижения поставленной цели необходимо было решить следующие задачи:

1) Изучить p-адические числа.

2)Исследовать отображения p-аических чисел на поле вещественных чисел.

3)Написать программу, осуществляющую работу генератора и отображающую полученные псевдослучайные числа на единичный куб.

Аналитическое исследование ГСПЧ бывает весьма непростой задачей, однако человек способен "на глаз" определять закономерности, что и позволяет оценить качество работы генератора 

Аналогом данного консольного приложения является графическое приложение Vorg, разработанное выпускником ВМК МГУ  Рябцевым В.С.
% \textbf{Объем и структура работы.} Работа состоит из~введения, четырех глав, заключения и~двух приложений. Полный объем диссертации составляет ХХХ~страница с~ХХ~рисунками и~ХХ~таблицами. Список литературы содержит ХХХ~наименований.

\clearpage
