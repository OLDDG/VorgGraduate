\chapter{Методы визуализации} \label{chapt2}
\section{Отображение $\Z_2 \rightarrow [0,1]$} \label{sect2_1}

Отображения Монна. Для решения поставленной задачи нам необходимо перевести числа из $\Z_2$ в отрезок [0,1]. Пусть число $x=\alpha_0\alpha_1\alpha_2\alpha_3\alpha_4\ldots$, тогда отображение определяется формулой:

\begin{equation}
    \zeta(x)=\sum_{i=1}^\infty \frac{\alpha_i}{p^i}
\end{equation}

Ряд является сходящимся при любом $x$, так как в знаменателе стоит степенная функция а числитель $\alpha_i$ может принимать только значения от 0 до 1.


Это отображение также сохраняет меру. Множество с мерой Хаара переводится во множество вещественных чисел с таким же значением меры Лебега. 
\vspace{5mm}

Другие отображения. Так же существует отображение, обратное отображению Монна, где старшими разрядами явлются последние разряды числа.

Есть и другой способ задать отображение  из $\Z_2$ в отрезок [0,1]. Если считать, что число $x$ содержит $k$ разрядов, то это отображение можно выразить формулой: 

\begin{equation}
    \phi(x)=\sum_{i=1}^\infty \frac{\alpha_{k-i}}{p^i}
\end{equation}



\section{Построение графиков в единичном кубе} \label{sect2_2}

В данной работе рассмотривается только случай $p$=2. Построение происходит в кубе $A=[0,1]\times[0,1]\times[0,1]$. Выбирается способ ввода данных, вручную или диапозон, для каждого из них задается размер $n$ регистра (длина слова),далее вычесления проводятся по (mod $2^n$). В случае выбора "диапозон" в автомат будут передаваться все числа от 0 до $2^n$ - 1 . Для каждой пары аргументов $x$, $y$ применяется закон$f(x, y)$. Также доступен выбор отображения Монна и обратное. Далее строится отображение a = $\zeta(x)$, b = $\zeta(y)$ и c = $\zeta(f(x,y))$

Таким образом, получается $2^n$ значений, которые в свою очередь являются точками на единичном кубе.

\if 0
\section{Построение в режиме Последовательность} \label{sect2_3}

В данном приложении доступен выбор режима "последовательность". При выборе этого режима пользователь также должен будет ввести свою функцию, размер регистра n и выбрать отображение: Монна или обратное ему. Далее необходимо ввести начальные значения $x_0$ и $y_0$ и количество итераций N. После всех введенных данных, будет N раз вызвана функция f($x_0$, $y_0$), f(f()), f(f(f())) ...
После всех вычислений будет получено N точек, которые будут перенесены на куб $A=[0,2^n]\times[0,2^n]\times[0,2^n]$. Результатом работы "хорошего" генератора будет картина, не содержащая явных последовательностей и повторений, схожая с картиной броуновского движения частиц.
Далее будут представлены примеры работы некоторых генераторов.
\fi

\clearpage

