\chapter{$P$-адические генераторы} \label{chapt1}

\section{Целые $p$-адические числа} \label{sect1_1}

Пусть задано простое число $p$ и множество $P=\{0,1,\ldots , p-1\}$. Целое p-адическое число $x$ может быть записано в виде степенного ряда $x=\alpha_0+p\alpha_1+\ldots+p^n\alpha_n\ldots$, где $\alpha_i\in P$, $n\geq0$. Во взаимно однозначное соотвествие такому ряду ставится бесконечная последовательность вида: 

$$\alpha_0 \alpha_1 \ldots \alpha_n \ldots, \quad \alpha_i \in P, i = 0,1,2 \ldots$$


$P$-адические числа во многом похожи  на $p$-ичные, но различаются тем, что в данном случае слева записываются младшие разряды, а старшие уже идут вправо. Такая запись чисел более удобная, так как последовательности являются бесконечными. Если все разряды числа старше $K$-го разряда нулые, то число будет записано в виде  $\alpha_0 \alpha_1 \ldots \alpha_k$, то есть:

\begin{equation}
   \alpha_0 \alpha_1 \ldots \alpha_k \Leftrightarrow \sum_{i=0}^k \alpha_i p^i
\end{equation}


\section{Арифметика в $\Z_p$} \label{sect1_2}

Для $p$-адических чисел  операции сложения, умножения и возведения в степень вводятся также, как и для обыкновенных целых чисел.

Суммой целых $p$-адических чисел $a = \alpha_0 \alpha_1$ \ldots $\alpha_n$ \ldots и $b = \beta_0 \beta_1$ \ldots $\beta_n$ \ldots\} называется целое $p$-адическое число, получаемое из них по правилам сложения «в столбик» в $p$-ичной системе счисления, если записать два числа одно под другим ($\beta_0$ под $\alpha_0$, $\beta_1$ под $\alpha_0$,
и т. д.).

Аналогичным образом определяются произведение и разность.

Вычитание всегда выполнимо, так как мы всегда можем "занять" единицу в следующем разряде числа (поскольку оно бесконечно). Для умножения на каждом шаге нам необходимо знать конечное число сложение, так что эта арифметическая операция также всегда выполняется. 

Множество $p$-адических чисел является группой относительно сложения, так как любое натуральное число представимо в виде целого $p$-адического числа.

Операция умножения не выводит нас из множества целых $p$-адических чисел, следовательно наше множество является кольцом ($\Z_p$).

\section{Отрицательные числа} \label{sect1_3}

Характерной особенностью $p$-адических чисел, является то, что запись отрицательного числа не имеет знака минус. Покажем это на примере числа -1 в $2$-адической системе: -1 будет представимо в виде бесконечного числа единиц: 11111 \ldots. Используем равенство -1+1=0. 
\vspace{5mm}

1 i i i i \ldots \\
+ \\

\underline{1\quad\quad\quad\quad}

0 0 0 0 0 \ldots

\vspace{5mm}

Единица переноса, появляющаяся после сложения первого же разряда, провоцирует перенос во всех последующих разрядах. В результате у нас получается бесконечная последовательность нулей. 

Аналогично можно получить запись любого отрицательного число -$d$ ($d \in \N$ ). Пусть $d =  \alpha_0 \alpha_1 \alpha_2 \alpha_3 \ldots \alpha_k$ , -$d = \beta_0 \beta_1 \beta_2 \beta_3 \ldots \beta_k$. Воспользуемся равенством $-d+d=0$. 

\vspace{5mm}

{ $\alpha_0 \alpha_1 \alpha_2 \ldots \alpha_n \ldots$ }\\
+ \\

\underline{$\beta_0 \beta_1 \beta_2 \ldots \beta_n \ldots$}

 0  0  0  \ldots  0 \ldots

\vspace{5mm}

Из равенств $\alpha_i + \beta_i + \sigma_i \equiv 0$ (mod $p$) где $\sigma_i$ означает наличие переноса в $i$-ом разряде ($\sigma_0 = 0$), можно поразрядно, начиная с $\alpha_0$, восстановить число $-d$ до какого угодно разряда, а также выявить закономерность, по которой чередуется $\beta_i$. 

\section{Перевод рационального числа в $p$-адическое} \label{sect1_4}

Другая особенность $\Z_p$ состоит в том, что некоторые рациональные числа также представимы в виде целых $p$-адических чисел. Если в несократимой дроби $\frac{r}{q}$ число $r$ целое, $q$ - натуральное, то эта дробь представима в виде $p$-адического числа, тогда и только тогда, когда знаменатель не должен быть кратен $p$. Это легко увидеть если записать умножение в "столбик". 

Воспользуемся равенством $\frac{r}{q} * q = r$. Пусть $\frac{r}{q} = \alpha_0 \alpha_1 \alpha_2 \alpha_3 \alpha_4 \ldots, q = \beta_0 \beta_1 \beta_2 \beta_3 \beta_4 \ldots$ и запишем процедуру в столбик:

\vspace{5mm}

$\alpha_0 \quad \alpha_1 \quad \alpha_2 \quad \alpha_3 \quad \alpha_4 \quad \ldots $\\
$\times$ \\

\underline{$ \beta_0 \quad \beta_1 \quad \beta_2 \quad \beta_3 \quad \beta_4 \quad \ldots$}

{$\alpha_0\beta_0 \alpha_1\beta_0 \alpha_2\beta_0 \alpha_3\beta_0 \alpha_4\beta_0 \ldots$ }

\quad\quad{$\alpha_0\beta_1 \alpha_1\beta_1 \alpha_2\beta_1 \alpha_3\beta_1 \ldots$ }

\underline{\quad\quad\quad\quad{$\alpha_0\beta_2 \alpha_1\beta_2 \alpha_2\beta_2 \ldots$ }}

{$\gamma_0 \quad \gamma_1 \quad \gamma_2 \quad \gamma_3 \quad \gamma_4 \quad \ldots$}


\vspace{5mm}

Из равенств $\alpha_i\beta_0 + \ldots + \alpha_0\beta_i + \sigma_i = \gamma_i$ (mod $p$) где $\sigma_i$ означает наличие переноса на $i$-ую столбцовую сууму разрядов, находим очередной разряд $\x_i$ (в каждый момент времени все предыдущие разряды уже известны).  


\section{Норма, расстояние и мера в $\Z_p$} \label{sect1_5}

В $\Z_p$ можно ввести норму.  Пусть $x$ - $p$-адическое число не равное нулю:  $$x=x_0 x_1 x_2 x_3 x_4 \ldots$$

Его $p$-адической
нормой называется число: 
$$||x||_p = \frac{1}{p^(ord_p x)} $$

Где $ord_p x$ = min\{$i : x_i \neq 0$\} - показатель максимальной степени $p$ делящей $n$. 

Если $x = 0$,то $||x||_p = 0$.


Удобно можно переписать тождества типа 
\begin{equation}
  \label{eq:equation1}
   a\equiv b (mod \quad p^k)
\end{equation}

Эквивалентная формулировка на языке норм будет выглядеть так: 

\begin{equation}
  \label{eq:equation2}
   ||a - b || \leq \frac{1}{p^k}
\end{equation}

Из тождества (\ref{eq:equation1}) делаем вывод о том, что $(a-b)$ делится на $p^k$, а это означает, что $ord_p (a - b) \leq p^k$. Из определения норм получаем  (\ref{eq:equation2}). Вспомнив определение символа $ord_p x$, замечаем, что в числах $a$ и $b$ как минимум младшие $k$ разрядов совпадают.

В $\Z_p$ можно ввести меру. Обычно рассматривается мера Хаара, которая определяется так:
\begin{equation}
  \label{eq:equation3}
    \mu(a + p^k \Z_p) = \frac{1}{p^k} , \quad  a \in \Z_p
\end{equation}

Формула $a + p^k \Z_p$ определяет множество чисел, у которых фиксированы младшие $k$ разрядов, а старшие могут изменяться во всевозможных вариациях. 

\section{Работа генератора} \label{sect1_6}

Автомату задано некоторое начальное состояние, а каждое последующее является функцией от предыдущего. Наиболее распространены генераторы для 2-адического случая.

Пусть нам дан регистр длиной $n$ разрядов. В начальный момент времени в регистре содержался числа $x_0$ и $y_0$, а в каждый $i$-ый момент времени - числа $x_i$ и $y_i$. Перебираютсявсе значения от 0 до $2^n$, для каждой новой пары ($x_{i+1},y_{i+1}$) вычисляем значение $f(x_{i+1},y_{i+1})$ где $f : \Z_{2^2} \rightarrow \Z_2$.

\if 0 
определим как функцию от их предыдущих значений $x_{i+1} = y_i,\quad y_{i+1} = f(x_i, y_i)$mod$(2^n)$, где $f : \Z_2 \rightarrow \Z_2$. Функция $f(x,y)$ называется \textit{законом рекурсии}.
\fi

Базовыми операциями являются арифметические и поразрядные логорифмические операции, которые естественным образом продолжаются до непрерывных функций на все множество целых 2-адических чисел.

Для реализации на компьютере наиболее естественным образом применим случай с $p$=2 и размером регистра, равным размеру какого-нибудь целочисленного типа, используемого в данной машине (например байт, слово, двойное слово и .т.д.). При этом функции могут содержать операции сложения, умножения на число, побитовые логарифмические операции, такие как AND, OR,XOR,NEG, а также сдвиги регистра влево или вправо. Все эти операции непосредственно представляют собой команды процессора и не требуют каких-либо дополнительных затрат на их программную реализацию. Единственная операция, которая потребует специальных процедур для их вычисления, это возведение в степень.

\clearpage

